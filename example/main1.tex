\documentclass{article}

\usepackage{caption}
\usepackage{subcaption}
\usepackage{tikz}
\usepackage{pgfplots}
\usepgfplotslibrary{groupplots}
\pgfplotsset{compat=1.18}
\usepackage{cleveref}
\usetikzlibrary{external}
\usetikzlibrary{external-autocache}

% \tikzset{
%   external-autocache/output dir = tmp,
%   external-autocache/input extensions={.tikz},
%   external-autocache/verbose = true
% }
\begin{document}
See \Cref{fig:control} for a theorist's struggle, and \Cref{subcap:pid,subcap:state} for system responses (expected: 2a, 2b).

\begin{figure}[h]
  \centering
  \begin{tikzpicture}
    % XKCD-style stick figure
    \draw (0,0) circle (0.3); % Head
    \draw (0,-0.5) -- (0,-1.5); % Body
    \draw (-0.3,-0.8) -- (-0.6,-1.2); % Left arm
    \draw (0.3,-0.8) -- (0.6,-1.2); % Right arm
    \draw (-0.3,-1.5) -- (-0.5,-2); % Left leg
    \draw (0.3,-1.5) -- (0.5,-2); % Right leg
    \node[align=center] at (1.5,-1) {My system’s\\oscillating\\into chaos!};
    \node at (0,0.5) {Control Theorist};
  \end{tikzpicture}
  \caption{Stick-figure control theorist battling an unstable system.}
  \label{fig:control}
\end{figure}

\begin{figure}[h]
  \centering
  \begin{tikzpicture}
  \begin{groupplot}[
      group style={
          group size=2 by 1,
          horizontal sep=2cm
        },
      width=5cm,
      height=4cm,
      xlabel={Time},
      ylabel={Output},
      domain=0:5,
      xmin=0, xmax=5,
      ymin=0, ymax=1.5,
      % XKCD-style font and simplicity
      font=\footnotesize,
      tick style={thin,black},
      axis line style={-},
      % Title
      title style={%
          at={(current axis.north west)},%
          inner sep=0,%
          anchor={south west},%
          font=\normalfont\footnotesize,
          color=black,
          text width=3cm,% this entry is necessary to make the title into a parbox, which is required by \subcaption{} as parent element
        }
    ]
    \nextgroupplot[
        title={
            \subcaption{
                PID: Overshoot Overshare
              }\label{subcap:pid}
          }
      ]
    \addplot[smooth,blue] {1-exp(-x)+0.5*exp(-x)*sin(deg(4*x))}; % PID step response with overshoot
    \node[
        font=\footnotesize
      ] at (4,1.2) {\(y(t)\)};
    \nextgroupplot[
        title={
            \subcaption{
                State-Space: Matrix Mayhem
              }\label{subcap:state}
          },
      ]
    \addplot[smooth,red] {1-exp(-2*x)*cos(deg(2*x))}; % State-space damped oscillation
    \node[
        font=\footnotesize
      ] at (4,1.2) {\(x(t)\)};
  \end{groupplot}
\end{tikzpicture}
  \caption{Control system responses.}
  \label{fig:equations}
\end{figure}

\end{document}